\section{Results and discussion}\label{sec:ResultsDiscussion}

\textit{Results in 2 parts.
1. part analyzise
2. part future steps}


\subsection{Human factors results}
Discuss how a high stable latency could be more acceptable than a low unstable latency due to info found in Theory, Human Factors.


Its important to include how humans act in different scenarios, as regardless of the systems, the remote operator will always be a constant high latency of atleast 500 - 1000 ms \cite{ouden2022}. And how they react to a possible system latency will also be crucial for the full remote operation.
\import{./Tables/}{HumanFactorResults}


\cite{brandenburger2023vqreactiontimertc}
Test of human factors and realiable communcations via 5g
"Limited literature on:" 
"Positive effect of bitrate on quaility", "Stalling has worse effect on QoExperience, if bitrate is higher"
"Higher frame rate not linked to information assimiliation, but increased user enjoyment"
Tested with three different levels of bitrate
1, 6, 24 Mbps at 5, 15, 25 FPS
Stimulus: Light signal, Distance marker
Measurements: Responce accuracy, Responce speed

Study 1:
Higher bitrate -> faster answears, more correct
Higher FPS -> same speed on answear, same amount of correct
Study 2:
Higher bitrate -> same speed on answears, more correct
Higher FPS -> same speed on answear, same amount of correct

Bitrate is more important (source 5 in PP: https://dl.acm.org/doi/abs/10.1145/2072298.2072351)
Stimulus type:
Distance marker signs where answeared faster and more correct than light signals
5000 -> 3000 ms speed, and 0.51 -> 0.58 and 0.61 -> 0.69 correct.
This is promosing new with the implementation of ERTMS as the only input stimulus in signalling threw camera and video stream will be signs as the remote control operatior will get the ETCS directly in the control room. 
After certain bitrate less helpful


\cite{jernberg2024}
"However, it also seems to be the case that remote operators adapt to the circumstances they find themselves in; for example, they drive with a safety margin reducing risks to their personal chosen limit and do not override the barriers of their own choice."

"The reaction time in H1 (a car cutting in into the ego lane) increased more for each latency condition than the offset in time that the manipulation created. This suggests that even an unperceived increase in latency (based on subjective ratings) is affecting the participants in a way that makes them less observant of their surroundings. Combined with a self-reported decline in performance and control as the latency increases, a conclusion could be that the added mental workload is turning the primary driving task into a distraction in itself, by forcing the operator to devote an unusual amount of attention just to maintain speed and lane position."



\subsubsection{Adaptation to stable latency.} % Må endre tittel
Jernberg(2024) 
"Fixed latency seems to be better than varied latency" (Davis et al., 2010, Gnatzig et al., 2013) 

Gorsich et al. (2018) found that a higher latency results in more inaccurate behavior, with a drastic decrease starting at a latency of 600 ms, and Gnatzig et al. (2013) found that a constant latency of 500 ms was unproblematic for drivers when the vehicle was steadily kept at 30 km/h on their track. 
\cite{jernberg2024}

(Neumier 2019)
Could not confirm that fixed latency resultet in any better result than varied latency.
Could not confirm "Kang et al" \cite{kang2018} who stated that fixed latency leads to better driving performance than varying. \cite{neumeier2019}
% Kang did not at all mention fixed vs variable latencies. 

In the event of offloading a truck with a crane, the participents where exposed to constant levels of latency, and showing that delaly of joystick all the way up to 480 ms was basicly irrelevant for QoE and the results of task done pr time. However a delay of 880 significantly reduced the performance overall by atleast halfing the rating on the scores (1-5).
\cite{brunnström2020}


\subsection{Comparing latency for different vehicles}

After reading threw an  reasearch paper on the topic of how latency treshold are set. It became apparent to me that very few have done enhanced research on this very topic for railway remote control. Papers such as XX, who mentions anything about the maximum latency before loosing performance reference sources of other vehicles and their tests of remote control. 
\import{./Tables/}{LatencyThresholdResults}

In the case of XX who reference Jernberg 2024 adn their findings of. Jernberg goes into details of the hazards and proxy hazards that the driver were facing. As mentionen in Section 4 "Related Work" \cite{jernberg2024}
How in control of the vehicle were you during the drive (1-5)
Baseline 3.7, 100ms 3.5, 200ms 2.9


The most referenced remote control latency effect paper, Neumeier's paper XX, talks about how participent leave the car lane significanlty more with higher latency, even tho with stable high latency. But:
"In the Parking scenario, even no differences for whatever latency could be revealed" \cite{neumeier2019}
Scenarios, was driving with turns, and one parking. No hazards except latency. 

\cite{fp2r2dato2024d41_2} % Need more information about bitsize + other
D41.2 - Testing reports \& assessment 
Results of the remote driving of tramways demonstrator. 
Image latency 
- G2G, (capturing processing, compression, transmission, reception, decompressing, displaying)
- Oslo to Berlin
- Two atomic clocks on phones.
- Measured to be 340 - 380 ms (Always under 400 ms)

\subsubsection{Calculation of latency}
The measurements and therefore the calculations of latency is not easily done. Since its a fluctuating measurement that varies alot depending on the components in the system used, the way to calculate various on what the personens want to find. 


Very difficult from related works:
Real time video latency: \cite{kaknjo2018videolatency}
Here, the time of the visual event in front of
the camera is denoted as T1 and the time when the event
was detected on the receiving end as T2. The start of frame processing is denoted as T3.
\textit{TVL =(T1 - T1)-(T2 -T3)=T3 -T1}


\subsection{Threshold for acceptable latency}
Maybe include calculations of speed of train, compare to the previous and discuss a posible latency treshhold. End in that research on spesificly train drivers on railway should be tested.

Use this report \cite{fp2r2dato2023d5_4}


\subsection{Tools and protocols for remote control}

\import{./Tables/}{ToolsFactorResults}


//Kanskje ikke ha ny tittel? Bare ha result og diskusjon om hverandre
\section{Discussion}
Discussion of the result from the measurements, what they could mean and possible use cases for the inforation gathered

\subsection{Theory v. Practical test}
Compare from the PreDraft of what expected result and hopes were, and discussing them with the information of what happend in the test and measurments.

\subsection{Future work}
- MJPEG vs H.264

- Offshore or other remote controlled vehicles that can shead light or contribute to remote control railway.

Differnt aspects of latency \textit{What is the difference}
Fixed, static v dynamic.

\subsection{Parameters for latency testing}
What parameters are we measuring. 
Latency, Jitter, Packet loss.

\subsection{Parameters for latency evaluation}
speed, distance, braking distance, reaction time, margin for error.

- Lateral Deviation
- Max Steering angle
- Out of Lane Ratio 
- Avg Speed 
- Acceleration 


\subsection{Calculations of parameters}