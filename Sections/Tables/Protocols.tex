\begin{itemize}
    \item \textbf{Internet Control Message Protocol (ICMP):} Supports latency measurement through Ping and Traceroute. Ping reports Round-Trip Time (RTT) by sending packets to a destination and measuring return time, while Traceroute identifies slow network segments by tracking each hop. These tools offer quick basic insights but cannot measure one-way delay due to lack of clock synchronisation.

    \item \textbf{One-Way Active Measurement Protocol (OWAMP):} Measures one-way latency using timestamped packets and synchronised clocks. It isolates directional delay, which is essential in remote operation where upstream delay (camera to operator) often dominates system performance.

    \item \textbf{Two-Way Active Measurement Protocol (TWAMP):} Extends OWAMP by measuring both directions independently. TWAMP provides structured two-way delay measurements with greater granularity than ICMP RTT and is widely used for validating network performance guarantees.

    \item \textbf{Real-Time Transport Protocol (RTP) timestamps:} Embeds timestamps into media packets, enabling measurement of latency from packet capture to display when combined with synchronisation (e.g., NTP or PTP). This makes RTP well suited for analysing video latency in teleoperation systems.

    \item \textbf{Network Time Protocol (NTP) / Precision Time Protocol (PTP):} These are synchronisation mechanisms used to enable accurate latency measurement rather than measurement tools themselves. NTP provides millisecond accuracy, while PTP enables microsecond-level precision, both of which are required to compute reliable one-way delays.

    \item \textbf{Time-Sensitive Networking (TSN) timing:} Provides deterministic timing behaviour in packet-switched networks. TSN ensures predictable delay and minimal jitter, improving the reliability of one-way and two-way latency measurements and making it relevant for real-time control environments.
    
    \item \textbf{Real-Time Streaming Protocol (RTSP):} A session-control protocol commonly used for video streaming. RTSP allows clients to request, pause, or resume video streams and is valued for predictable buffering behaviour, making delay easier to analyse.

    \item \textbf{Web Real-Time Communication (WebRTC):} A browser-based framework for low-delay peer-to-peer media transport. Uses dynamic path negotiation and congestion control, which can reduce delay but also introduce variability. Useful for real-time applications where minimal setup time is needed.

    \item \textbf{Real-Time Transport Control Protocol (RTCP):} Works alongside RTP to provide feedback on network conditions such as jitter, packet loss, and delay. Enables adaptive streaming and is essential when evaluating how delay evolves under degraded network performance.

    \item \textbf{Transmission Control Protocol (TCP):} A reliable, connection-oriented protocol that ensures packets arrive in order through retransmissions. TCP is not ideal for delay-sensitive control due to its congestion-control behaviour, but its stability makes it useful for evaluating baseline delay or control-signal reliability.

    \item \textbf{User Datagram Protocol (UDP):} A connectionless protocol with no retransmissions or ordering guarantees. UDP provides minimal transport overhead, making it suitable for real-time video and control testing. Its lack of reliability mechanisms also reveals how systems behave under packet loss.
\end{itemize}
