\begin{itemize}
  \item \textbf{Level 0:} Applies to trains equipped for ETCS, but there is no ETCS trackside. Effectively going back to regular control and legacy signalling.
  \item \textbf{Level STM:} Applies to trains equipped for ETCS, but runs on tracks with national system with ATP. Allowing ETCS to interface for the ATP
  \item \textbf{Level 1:} Spot transmission threw Eurobalises providing intermittent movement authorities and speed control while legacy signalling remain in place as shown in Figure \ref{fig:ETCSlv1}.
  \begin{figure}[H]
    \centering
    \includegraphics[width=0.6\textwidth]{Images/ETCS lv1.png}
    \caption[ETCS Level 1]{Train following regular signal with assisted speed and position with balises}
    \label{fig:ETCSlv1}
    \source{\cite{eu_etcs_levels}}
  \end{figure}
  \item \textbf{Level 2:} Continuous radio exchange to Radio Block Centre (RBC), typically via GSM-R or a successor. Movement authority is provided by the RBC. Eurobalises, if used, are primarily for precise positioning. Legacy signalling system are no longer needed and optional as shown in Figure \ref{fig:ETCSlv2}.
  \begin{figure}[H]
    \centering
    \includegraphics[width=0.6\textwidth]{Images/ETCS lv2.png}
    \caption[ETCS Level 2]{Train operating speed and signal with RBC threw GSM-R signal}
    \label{fig:ETCSlv2}
    \source{\cite{eu_etcs_levels}}
  \end{figure}
  \cite{eu_etcs_levels}.
\end{itemize}