\begin{table}[h!]
\centering
\begin{tabular}{|c|p{10cm}|}
\hline
\textbf{GoA} & \textbf{Meaning / operator role} \\
\hline
GoA~0 & On-sight, manual operation without automatic protection. \\
\hline
GoA~1 & Atomatic Train Protection (ATP) Manual driving with assisted protection rutines. Human driver performs traction, braking and door tasks while safety limits are done automatic. That includes track speed, safe routing and safe separation. \\
\hline
GoA~2 & Semi-automated (STO). ATO handles start/stop and trajectory control between stations; a driver remains onboard for door operation, obstacle response and degraded mode handling. \\
\hline
GoA~3 & Driverless (DTO). No driver needed for normal operation. Staff may be on board for passenger assistance and emergencies. ATO handle operational tasks including avoiding collision with obstacles and persons. \\
\hline
GoA~4 & Unattended Train Operation (UTO). Fully automated operation without staff onboard. Remote supervision and controls are required for special incidents. \\
\hline
\end{tabular}
\caption{Grades of Automation (GoA), summary based on IEC and industry sources.}
\label{tab:goa}
\end{table}