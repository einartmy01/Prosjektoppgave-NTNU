\section{Related Work}

Experiences from other similar fields, remote controlled cars, drones, cranes and other remote controlled trains. These experiences and papers tests viable aspects for the project of ATO, mainly focusing on the remote controlled 

\import{./04 Related Work Controls}{Train Control}

\import{./04 Related Work Controls}{Car Control}


Going to hold on Sato and Nakamura too see if needed
\subsubsection{Sato 2021}
%https://ieeexplore.ieee.org/document/9575817
Base of 150 ms
buffer time of 150, 200, 400. (ignoringing base)

Showing latency imporved performance on every latency
The improvement increased as well for higher latency
150     200     400
14->3,  8->6,   7->6

\subsubsection{Nakamura 2021}
%https://ieeexplore.ieee.org/abstract/document/9622069
Samme som de andre, sim med lagt på latency.



\subsubsection{Jernberg 2024}
\cite{jernberg2024}
- Evaluates Latency by performance, (people) Sim

- Voysys
- G2G
- Average delay of 88.8 ms and added conditions of +100 -> 188 ms and +200 -> 288 ms
- 50km/h and 70km/h (Try to keep speedlimits)
- driver was not given a latency

+100 and +200 was chosen because of Neumeier result.
Reference Neumeier et al. (2019) stated that 300 ms might be manageable for trained operators but in some conditions during their simulator study there were tendencies that even smaller latencies affected the performance of the operator negatively.

That was:
H1/P1: Car pulling over into your lane.
H2/P2: Car crossing from opposing lane threw your lane.
H3/P3: Car with "vikeplikt" does not stop in crossing.
H4/P4: Child runs into traffic from behind a bus.
H5/P5: Bicycal in lane that driver needs to pass in oppsing lane.


According to Jernberg \cite{jernberg2024}, when performing a study with more naturalistic driving scenarios, speed and type of task is significant for results as well as latency. 
They also adapt to cerumstances, adjusting speed and safetymargins.


\subsubsection{Neumier 2019}
\cite{neumeier2019}
- Evaluate latency by performance, (people) Sim

 - Round-Trip Time (RTT)
 - Average delay of 67 ms and added conditions of +100 +300 +500
 - Was given the ca. latency

Talks about how participent leave the car lane significanlty more with higher latency, even tho with stable high latency. But:
"In the Parking scenario, even no differences for whatever latency could be revealed"
Scenarios, was driving with turns, and one parking. No hazards except latency.


\subsubsection{Kang 2018}
\cite{kang2018}
- Evaluate latency by perfomance, (different tools) IRL, 

- 3 different resulutions (320x240, 640x480, 1280x960)
- 3 different bitrate (0.5Mbps, 1Mbps, 4Mbps)
- LTE and WiFi
- Video and camerea catching timestamps



\subsection{Drone Control}
%Ikke sjekk Emilia sin, hun har ikke kilder

\subsubsection{N. González 2023}
\cite{gonzález2023}
- Evaluation tools for latency, 
- People tested

URLLC
4K quality res 1080 x 720, at 30 FPS
H.264 encoded

LTE server, LTE direct, WiFi -> avg packet delay =  500 ms, 42 ms 4 ms %Er et bilde i filutforsker hvis trengs
Can activate low latency mode. 
Connection requirments of 100 ms for video streaming, set by 3GPP TS 22.829 for unmanned aerial vehicles
Found added latency of around 180 start causing lower MOS / QoE, and steady deteriation making it 1, lowest grade at around 460 ms
concludes with a e2e of 250 ms is vaiable for service usability.

\subsubsection{Larsen 2022}
\cite{larsen2022}
- Evaluation latency on perfomance, (tools) IRL

5G URLLC network
H.264

Le2e = Lprop agation + Lproc essing + Lser ialisation
Lprop = distance / v in medium
Lser = S datasize / R transmission rate

Le2e = nLproc + (n+1)Lser + Lprop + LQ 
n = switches along the network
n+1 = number of links
LQ = queing latency.

0.5 Mbps video rate in uplink and a 60 Hz update rate in downlink. Further, we assume that the higher quality video for inspection require 8 Mbps.


\subsubsection{Böhmer 2020}
\cite{böhmer2020}
- Evaluation latency on perfomance, (tools) IRL

Predictably Reliable Real-time Transport (PRRT) protocol [A. Schmidt, “Cross-layer latency-aware and -predictable data communication 2019]
The Crazyflie is controlled by Bitcraze's application layer protocol called Crazy Real-Time Protocol (CRTP)

Rasberry Pi including WiFi 2.4GHz due to Rasberry Pi constraints
timestamps by controller to drone: 
tp1 -> packet1 -> drone
tr1 <- response1 <- drone
tp2 -> packet2 -> drone
tr2 <- response2 <- drone

the Crazyradio communication path using the traditional radio link
the PRRT communication path with the Python bridge, and
the PRRT communication path with the Rust bridge.

IPT = tp2 - tp1 (Time between packets)
RTT = tr1 - tp1 (Round-trip time)

\subsection{Crane Control}
\subsubsection{Brunnström 2020}
\cite{brunnström2020}
- Evaluate latency on performance, (Humans) Sim


To study QoE. 
- VR.
- 270 angle HMD video threw 4 cameras on crane.
Mission is to offload a truck full of logs.
Tested with diffrent delays for display and joystick. Baseline was 25 ms for display and 80 ms for joystick.
- Display, 5, 10, 20, 30 -> 25, 30, 35, 45, 55
- Joystick, 10, 20, 50, 100, 200, 400, 800 -> 80, 90, 100, 130, 180, 280, 480, 880

Comfort of the subject was not affected by joystick delay, but the display delayed had an negative effect of Comfort quality. Why this could be discussed alot. Might have to do with VR and a more moving image. 
480 ms gave a mild reduction in effect and quality of the work at hand but at 880 it was a major decrease in effect and operatbility. 


\begin{comment}
\subsection{Ethics}
Who is responsible. Fully automated, or remote driver.
What obligation do we have in a project like this.

https://www.tandfonline.com/doi/full/10.1080/01441647.2020.1862355


\subsection{Cybersecurity}
How we can protect the system from attacks. What regulations are in place. 
"[CYB] CLC/TS 50701:2023, Railway applications - Cybersecurity" used in \cite{fp2r2dato2023d5_4}
\end{comment}





