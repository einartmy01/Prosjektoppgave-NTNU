\section{Related Work}

Experiences from other simialar fields, remote controlled cars, drones, cranes and other remote controlled trains. These experiences and papers tests viable aspects for the project of ATO, mainly focusing on the remote controlled 

\subsection{Train Control}
%Usikker
As mentionted in ERA's ERTMS. Commission Regulation (EU) 2016/919 of 27 May 2016 on the technical specification for 
interoperability relating to the "control-command and signalling" subsystems of the rail system in the European Union \cite{era_ertms}

\subsubsection{ATO-Cargo Project}
The ATO-Cargo project, led by the German Aerospace Center (DLR) in cooperation with DB Cargo AG, Digitale Schiene Deutschland (DSD), and ProRail B.V., focuses on developing and testing highly automated technologies for freight trains. The goal is to enhance rail freight efficiency by optimizing speed profiles, improving route utilization, and increasing competitiveness with road transport \cite{dlr2025atocargo}.

A key component of the project is the integration of an Automatic Train Operation (ATO) unit on locomotives in combination with the European Train Control System (ETCS) Level 2. This setup allows for real-time automation while maintaining human oversight. In case of system malfunctions or degraded operation, human operators at a Remote Supervision and Control Centre (RSC) can take over tasks such as remote monitoring, diagnosis, and manual control. 

The project also emphasizes human factors engineering, ensuring that the RSC is ergonomically designed for operator efficiency and safety. For this project, researchers have employ virtual reality tools to simulate realistic control room environments and train personnel for future remote supervision tasks. Tests are being conducted on Betuweroute, a freight only railway, linking Rotterdam and the Ruhr region to validate the technical and operational readiness of this automation concept. The ultimate goal is to establish a European reference model for automated freight train operation \cite{dlr2025atocargo}.

\subsubsection{Jürgensen 2025}
\cite{jürgensen2025rcrailvehicles}
- Finds actually latency in project
- Uses Car threshold to evaluate


Is a project in "Remote Control for Rail Vehicles" where they test a remote control train for a short track from "X" to "X" in Germany. Here they do "this" and found "that".

Says that at 300ms, you get loss of performance, and at 1000ms the delay becomes unfeasible.
And references: "Design and Evaluation of Remote Driving Architecture on 4G and 5G Mobile Networks" (Ouden, 2022) \cite{ouden2022}
Which references (Lane, 2002) and (Neumier, 2019)




\subsubsection{Kozarevic 2025}
- Finds actually latency in project
- Uses Car / Drone threshold to evaluate
%emilia prosjektoppgave
Talks about the dangers of latency in high speed vehicles. Comparing it to drone operations.
Reference drone and says 100 ms
Reference Chen?, and Neumier, 170 ms and 300 ms have minimal impact on remote operators.

\subsubsection{Mejías 2024}
\cite{mejias2024}
- Evaluate latency depending on parameters / tools

Compare:
- RTSP
- WebRTC Web Real time communcation
as their Real-time Transport Protocol (RTP) protocol.

- E2E
- H.264

Methodology for latency measurment.
Network Time Protocol (NTP) is necesarry to synchronize sender and receiver.
Server obtains the TS1 when image is captured. Adds it to the RTP packets generated after the encoding. Player retrives the timestamp (TS1) from the RTP packets and compares it with the current time TS2 when the image is beeing displayed. To do this, you must retrive it from the package before the decoder and comapre it with the image comming out of the decoder. 

1 Capture: the camera captures an image together with the timestamp. The timestamp is added to the metadata of the image.
2 Encoding: the image is encoded into a H.264 bitstream. The metadata is maintained unaltered along the encoding process.
3 Encapsulation: the H.264 video stream is encapsulated into RTP payload. The capture timestamp is extracted from the metadata and added to the RTP header. For this, it is required both the RTP standard header and its RFC 8286 extension.
4 Sender: RTP packets are sent on the communication channel. In the case of RTSP, the player opens a connection with the sender. For WebRTC, a negotiation between the sender and receiver is performed through the signaling server to determine the communication route.

The player receives the RTP packets through RTSP or WebRTC and calculates the latency:
1 Receiver: it receives the RTP packets through the channel established with the media server.
2 Decapsulation: the original H.264 content is extracted from the RTP payload. In addition, the timestamp contained in the RTP header is extracted and added as metadata of the H.264 content.
3 Decoding: the H.264 content is decoded to retrieve the uncompressed image. The metadata is maintained unaltered along the decoding process.
4 Displaying: the image is displayed. Moreover, the timestamp is extracted from the metadata and subtracted from the current time to obtain the End-to-End latency. This is shown to the remote driver, who will consider it during the operations.

BITRATE 
Change bitrate regarding quality of output (jitter or packet loss). Bitrate vary between 5Mbps, 3.5Mbps, 2Mbps.
A change of 2\% packet loss and 500Hz / 1000Hz jitter.


implementation.
GStreamer framework. (Open source)
Pylon source from Basler element that capture of camera images and timestamps.
H.264 NVidia en/de coder. Provided by NVIDIA graphics cards. It is the key to enable bitrate adaption.
RTP H.264 pay/depay. For packaging encoded video signal into RTP packets, and RTP includes timestamp in header.

WebRTCbin. Allows communcation via WebRTC, peer2peer, must connect to signaling server resposible for negotiation.
RTSP server/client. Manage connection and send/recive data.

Camera, Media server on Jetson Xavier (Either WebRTC or RTSP), Network equipment (switch or laptop simulating a router, allowing to evaluate against bandwith degradation). Computer as player and reciver.


Results
E2E, time after capture to the time before display
S2S, time in front of camera to time displayed on player image 

Measurments for each camera and alternating available bandwidth.
When enough bandwidth results in 150 ms S2S and <75 ms E2E 
Bandwidth of <=10 results in 570 ms - 1000 ms or pixelation freezing in both S2S and E2E

RTSP
Difference in S2S and E2E is approx 70 ms - 100 ms which is image capturing and displaying.

WebRTC is faster E2E but not S2S

RTSP with rate control
Allows the bandwidth to go past 7 Mbps that was issue before, although with high latency. 
Adjust itself back up again. Also we can see a shift in latency between latency when increased bitsize of video.  

\subsubsection{FP2R2Dato, EuropesRail 2023 D5.4}
\cite{fp2r2dato2023d5_4}
- Uses Drone latency to evaluate threshold

Based on drone latency. Enhensive calculations of latency to match to train. 


\subsubsection{FP2R2Dato, EuropesRail 2024 D41.2}
D41.2 - Testing reports \& assessment 
Results of the remote driving of tramways demonstrator. 
\cite{fp2r2dato2024d41_2}
- Finds actually latency

Image latency 
- G2G, (capturing processing, compression, transmission, reception, decompressing, displaying)
- Oslo to Berlin
- Two atomic clocks on phones.
- Measured to be 340 - 380 ms (Always under 400 ms)


Auxiliary circuit tests,
- Driver safety
- Remote wake up
- R driving loop
- R control commands
- R ...
Static functional tests,
- Start Tram, CERES,
- CERES do step 1.2.3\dots

Dynamis functional tests,
- CERES Local brakte test
- CERES Remote Brakte test 
- C Drive 5km/h 
- C Drive 100\% dont break max speed 
- C loose communication.
- C DSD brake sequence
- C Local driver break priority over remote.

Reaction Time,
reduction in time needed to perform spesific tram operations, fleet management and preparation, start-up and shut-down procedures, maintenance tasks, shunting. 
Time reduction in these processes are efficiency improvments.





\subsubsection{Brandernburger 2023}
\cite{brandenburger2023vqreactiontimertc}
- Evaluates latency by performance, (human) IRL


Test of human factors and realiable communcations via 5g
"Limited literature on:" 
"Positive effect of bitrate on quaility", "Stalling has worse effect on QoExperience, if bitrate is higher"
"Higher frame rate not linked to information assimiliation, but increased user enjoyment"

Tested with three different levels of bitrate
1, 6, 24 Mbps
5, 15, 25 FPS
Stimulus
Light signal, Distance marker

Measurements
Responce accuracy
Responce speed

Study 1:
Higher bitrate -> faster answears, more correct
Higher FPS -> same speed on answear, same amount of correct

Study 2:
Higher bitrate -> same speed on answears, more correct
Higher FPS -> same speed on answear, same amount of correct

Bitrate is more important (source 5 in PP: https://dl.acm.org/doi/abs/10.1145/2072298.2072351)

Stimulus type:
Distance marker signs where answeared faster and more correct than light signals
5000 -> 3000 speed, and 0.51 -> 0.58 and 0.61 -> 0.69

This is promosing new with the implementation of ERTMS as the only input stimulus in signalling threw camera and video stream will be signs as the remote control operatior will get the ETCS directly in the control room. 


After certain bitrate less helpful


\subsection{Car Control}

\subsubsection{Sato 2021}
%https://ieeexplore.ieee.org/document/9575817
Base of 150 ms
buffer time of 150, 200, 400. (ignoringing base)

Showing latency imporved performance on every latency
The improvement increased as well for higher latency
150     200     400
14->3,  8->6,   7->6

\subsubsection{Nakamura 2021}
%https://ieeexplore.ieee.org/abstract/document/9622069
Samme som de andre, sim med lagt på latency.

\subsubsection{Ouden 2022}
\cite{ouden2022}
- Evaluates Latency by performance, (people) IRL  

- Evaluate Latency threshold on SIMulation

- 4G and 5G 
- 4 times 120 angle camera
- H.264
- Split latency up into Control and video and does
- min, mean, 95\%ile, max latency in ms

- Speed of 10, 20, 30 and 40km/h
100 manual test runs for benchmark
180 runs of RC with 4G
300 runs of RC with 5G 
One Trip Latency.” Every unit
was time synchronized with a GPS-PPP source.
Packets logged using tcpdump.


Includes a test that results in at 300ms, you get loss of performance, and at 1000ms the delay becomes unfeasible.

Did not find resonable difference in the latencies of straight acceleration test. 

\subsubsection{Jernberg 2024}
\cite{jernberg2024}
- Evaluates Latency by performance, (people) Sim

- Voysys
- G2G
- Average delay of 88.8 ms and added conditions of +100 -> 188 ms and +200 -> 288 ms
- 50km/h and 70km/h (Try to keep speedlimits)
- driver was not given a latency

+100 and +200 was chosen because of Neumeier result.
Reference Neumeier et al. (2019) stated that 300 ms might be manageable for trained operators but in some conditions during their simulator study there were tendencies that even smaller latencies affected the performance of the operator negatively.

That was:
H1/P1: Car pulling over into your lane.
H2/P2: Car crossing from opposing lane threw your lane.
H3/P3: Car with "vikeplikt" does not stop in crossing.
H4/P4: Child runs into traffic from behind a bus.
H5/P5: Bicycal in lane that driver needs to pass in oppsing lane.


According to Jernberg \cite{jernberg2024}, when performing a study with more naturalistic driving scenarios, speed and type of task is significant for results as well as latency. 
They also adapt to cerumstances, adjusting speed and safetymargins.
\subsubsection{Kaknjo 2018}
\cite{kaknjo2018videolatency}
- Evaluate latency by perfomance, (different tools) IRL, 

- G2G
- Time stamps
- H.264
- MJPEG
- RTSP (Real Time Streaming protocol)
- TCP/UDP

Found MJPEG to be 300 ms lower latency than H.264. However it found H.264 to demand less bandwitdh 50-380Kbps as it compresses more enhensive than MJPEG 4.6-5Mbps.
Found detoriation in performance in latencies above 300 ms and increase in errors during control for latencies larger than 500 ms.

\subsubsection{Neumier 2019}
\cite{neumeier2019}
- Evaluate latency by performance, (people) Sim

 - Round-Trip Time (RTT)
 - Average delay of 67 ms and added conditions of +100 +300 +500
 - Was given the ca. latency

Talks about how participent leave the car lane significanlty more with higher latency, even tho with stable high latency. But:
"In the Parking scenario, even no differences for whatever latency could be revealed"
Scenarios, was driving with turns, and one parking. No hazards except latency.


\subsubsection{Kang 2018}
\cite{kang2018}
- Evaluate latency by perfomance, (different tools) IRL, 

- 3 different resulutions (320x240, 640x480, 1280x960)
- 3 different bitrate (0.5Mbps, 1Mbps, 4Mbps)
- LTE and WiFi
- Video and camerea catching timestamps



\subsection{Drone Control}
%Ikke sjekk Emilia sin, hun har ikke kilder

\subsubsection{N. González 2023}
\cite{gonzález2023}
- Evaluation tools for latency, 
- People tested

URLLC
4K quality res 1080 x 720, at 30 FPS
H.264 encoded

LTE server, LTE direct, WiFi -> avg packet delay =  500 ms, 42 ms 4 ms %Er et bilde i filutforsker hvis trengs
Can activate low latency mode. 
Connection requirments of 100 ms for video streaming, set by 3GPP TS 22.829 for unmanned aerial vehicles
Found added latency of around 180 start causing lower MOS / QoE, and steady deteriation making it 1, lowest grade at around 460 ms
concludes with a e2e of 250 ms is vaiable for service usability.

\subsubsection{Larsen 2022}
\cite{larsen2022}
- Evaluation latency on perfomance, (tools) IRL

5G URLLC network
H.264

Le2e = Lprop agation + Lproc essing + Lser ialisation
Lprop = distance / v in medium
Lser = S datasize / R transmission rate

Le2e = nLproc + (n+1)Lser + Lprop + LQ 
n = switches along the network
n+1 = number of links
LQ = queing latency.

0.5 Mbps video rate in uplink and a 60 Hz update rate in downlink. Further, we assume that the higher quality video for inspection require 8 Mbps.


\subsubsection{Böhmer 2020}
\cite{böhmer2020}
- Evaluation latency on perfomance, (tools) IRL

Predictably Reliable Real-time Transport (PRRT) protocol [A. Schmidt, “Cross-layer latency-aware and -predictable data communication 2019]
The Crazyflie is controlled by Bitcraze's application layer protocol called Crazy Real-Time Protocol (CRTP)

Rasberry Pi including WiFi 2.4GHz due to Rasberry Pi constraints
timestamps by controller to drone: 
tp1 -> packet1 -> drone
tr1 <- response1 <- drone
tp2 -> packet2 -> drone
tr2 <- response2 <- drone

the Crazyradio communication path using the traditional radio link
the PRRT communication path with the Python bridge, and
the PRRT communication path with the Rust bridge.

IPT = tp2 - tp1 (Time between packets)
RTT = tr1 - tp1 (Round-trip time)

\subsection{Crane Control}
\subsubsection{Brunnström 2020}
\cite{brunnström2020}
- Evaluate latency on performance, (Humans) Sim


To study QoE. 
- VR.
- 270 angle HMD video threw 4 cameras on crane.
Mission is to offload a truck full of logs.
Tested with diffrent delays for display and joystick. Baseline was 25 ms for display and 80 ms for joystick.
- Display, 5, 10, 20, 30 -> 25, 30, 35, 45, 55
- Joystick, 10, 20, 50, 100, 200, 400, 800 -> 80, 90, 100, 130, 180, 280, 480, 880

Comfort of the subject was not affected by joystick delay, but the display delayed had an negative effect of Comfort quality. Why this could be discussed alot. Might have to do with VR and a more moving image. 
480 ms gave a mild reduction in effect and quality of the work at hand but at 880 it was a major decrease in effect and operatbility. 

\subsection{Ethics}
Who is responsible. Fully automated, or remote driver.
What obligation do we have in a project like this.

https://www.tandfonline.com/doi/full/10.1080/01441647.2020.1862355


\subsection{Cybersecurity}
How we can protect the system from attacks. What regulations are in place. 
"[CYB] CLC/TS 50701:2023, Railway applications - Cybersecurity" used in \cite{fp2r2dato2023d5_4}






