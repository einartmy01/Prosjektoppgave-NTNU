\section{Related Work}

Experiences from other similar fields, remote controlled cars, drones, cranes and other remote controlled trains. These experiences and papers tests viable aspects for the project of ATO, mainly focusing on the remote controlled 

\import{./04 Related Work Controls}{Train Control}

\import{./04 Related Work Controls}{Car Control}


\textit{Going to hold on Sato and Nakamura too see if needed}
\subsubsection{Sato 2021}
%https://ieeexplore.ieee.org/document/9575817
Base of 150 ms
buffer time of 150, 200, 400. (ignoringing base)

Showing latency imporved performance on every latency
The improvement increased as well for higher latency
150     200     400
14->3,  8->6,   7->6

\subsubsection{Nakamura 2021}
%https://ieeexplore.ieee.org/abstract/document/9622069
Samme som de andre, sim med lagt på latency.


\import{./04 Related Work Controls}{Drone Control}




\subsubsection{Böhmer 2020}
\cite{böhmer2020}
- Evaluation latency on perfomance, (tools) IRL

Predictably Reliable Real-time Transport (PRRT) protocol [A. Schmidt, “Cross-layer latency-aware and -predictable data communication 2019]
The Crazyflie is controlled by Bitcraze's application layer protocol called Crazy Real-Time Protocol (CRTP)

Rasberry Pi including WiFi 2.4GHz due to Rasberry Pi constraints
timestamps by controller to drone: 
tp1 -> packet1 -> drone
tr1 <- response1 <- drone
tp2 -> packet2 -> drone
tr2 <- response2 <- drone

the Crazyradio communication path using the traditional radio link
the PRRT communication path with the Python bridge, and
the PRRT communication path with the Rust bridge.

IPT = tp2 - tp1 (Time between packets)
RTT = tr1 - tp1 (Round-trip time)

\subsection{Crane Control}
\subsubsection{Brunnström 2020}
\cite{brunnström2020}
- Evaluate latency on performance, (Humans) Sim


To study QoE. 
- VR.
- 270 angle HMD video threw 4 cameras on crane.
Mission is to offload a truck full of logs.
Tested with diffrent delays for display and joystick. Baseline was 25 ms for display and 80 ms for joystick.
- Display, 5, 10, 20, 30 -> 25, 30, 35, 45, 55
- Joystick, 10, 20, 50, 100, 200, 400, 800 -> 80, 90, 100, 130, 180, 280, 480, 880

Comfort of the subject was not affected by joystick delay, but the display delayed had an negative effect of Comfort quality. Why this could be discussed alot. Might have to do with VR and a more moving image. 
480 ms gave a mild reduction in effect and quality of the work at hand but at 880 it was a major decrease in effect and operatbility. 


\begin{comment}
\subsection{Ethics}
Who is responsible. Fully automated, or remote driver.
What obligation do we have in a project like this.

https://www.tandfonline.com/doi/full/10.1080/01441647.2020.1862355


\subsection{Cybersecurity}
How we can protect the system from attacks. What regulations are in place. 
"[CYB] CLC/TS 50701:2023, Railway applications - Cybersecurity" used in \cite{fp2r2dato2023d5_4}
\end{comment}





