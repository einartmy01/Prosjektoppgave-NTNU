\section{Related Work}

\subsection{Other Experiences}
Experiences from other simialar fields, self driving cars, drones, maybe other remote controlled trains in different countries

\subsubsection{Train Control}
In accordance to UITP, of all fully automated metro lines, 87\% use CBTC (communications based train control) \cite{uitp2018metro}

As mentionted in ERA's ERTMS. Commission Regulation (EU) 2016/919 of 27 May 2016 on the technical specification for 
interoperability relating to the ‘control-command and signalling’ subsystems of the rail system in the European Union \cite{era_ertms}

\subsubsection{ATO-Cargo Project}
The ATO-Cargo project, led by the German Aerospace Center (DLR) in cooperation with DB Cargo AG, Digitale Schiene Deutschland (DSD), and ProRail B.V., focuses on developing and testing highly automated technologies for freight trains. The goal is to enhance rail freight efficiency by optimizing speed profiles, improving route utilization, and increasing competitiveness with road transport \cite{dlr2025atocargo}.

A key component of the project is the integration of an Automatic Train Operation (ATO) unit on locomotives in combination with the European Train Control System (ETCS) Level 2. This setup allows for real-time automation while maintaining human oversight. In case of system malfunctions or degraded operation, human operators at a Remote Supervision and Control Centre (RSC) can take over tasks such as remote monitoring, diagnosis, and manual control. 

The project also emphasizes human factors engineering, ensuring that the RSC is ergonomically designed for operator efficiency and safety. For this project, researchers have employ virtual reality tools to simulate realistic control room environments and train personnel for future remote supervision tasks. Tests are being conducted on Betuweroute, a freight only railway, linking Rotterdam and the Ruhr region to validate the technical and operational readiness of this automation concept. The ultimate goal is to establish a European reference model for automated freight train operation \cite{dlr2025atocargo}.

Sjekk T-teknikk
\subsubsection{Car Control}
//Distraction lantecy?

\subsubsection{Drone Control}
//Sjekk Emilia sin

\subsubsection{Crane Control?}

\subsubsection{Offshore Control}




//Usikker på om dette skal være med
\subsection{Remote Control}
How remote control works, how it is set up. Why its relevant for this project.

\subsection{Human factors}
All the ways human error can effect the results from the tests.
All the ways human control needs to be adjusted for in acceptance levels.

How humans adapt to stable v unstable latencies.



"Fixed latency seems to be better than varied latency" (Davis et al., 2010, Gnatzig et al., 2013) 

Gorsich et al. (2018) found that a higher latency results in more inaccurate behavior, with a drastic decrease starting at a latency of 600 ms, and Gnatzig et al. (2013) found that a constant latency of 500 ms was unproblematic for drivers when the vehicle was steadily kept at 30 km/h on their track. Neumeier et al. (2019) stated that 300 ms might be manageable for trained operators but in some conditions during their simulator study there were tendencies that even smaller latencies affected the performance of the operator negatively.
Jernberg(2024) 
\cite{jernberg2024latency}
- Voysys
- G2G
- Average delay of 88.8 ms and added conditions of +100 -> 188 ms and +200 -> 288 ms
- 50km/h and 70km/h


\subsection{Acceptance levels}
\textit{Sjekk master til Emilia om hva hun fant av regler}
Connect up to what other projects have set as acceptance level, and why.
What we can accept and why we choose these limits.

\subsection{Ethics}
Who is responsible. Fully automated, or remote driver.
What obligation do we have in a project like this.


\subsection{Cyber Security}
How we can protect the system from attacks. What regulations are in place. 

\subsection{Calculation of latency}
The math behind calculations of latency. How to measure most precisely and what errors  we find in the calculations.

Very difficult from related works:
Real time video latency: \cite{kaknjo2018videolatency}
Here, the time of the visual event in front of
the camera is denoted as T1 and the time when the event
was detected on the receiving end as T2. The start of frame processing is denoted as T3.
\textit{TVL =(T1 - T1)-(T2 -T3)=T3 -T1}



