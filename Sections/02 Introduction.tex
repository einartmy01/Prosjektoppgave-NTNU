\section{Introduction}
Introduction to the paper and theories that are going to be used.

\subsection{Purpose}
\textit{What is the purpose of this paper. Why are we doing it.}
Find a way to effectively measure, validate and evaluate latency for remote train operation systems. 
Establish acceptable latency thresholds that ensure safety and performance. Identify key factors influencing latency and propose optimization strategies.

\subsubsection{Research Questions}
\textit{What are we hoping to answear and/or achieve during this paper}
To achieve the purpose of this paper, the research questions listed below was created to help. 

\begin{enumerate}
    \item RQ1: How does other industries set threshold for latency?
    \item RQ2: What parameters influence latency in remote train operation systems?
    \item RQ3: How can we test different video streaming protocols and their impact on real-time performance of remote train operation systems?
\end{enumerate}

% ===========================
% Body: Sections (paste into your .tex)
% ===========================

\subsection{Digitalization in Railway}
Railway digitalization evolved from early computer-assisted signalling and centralized traffic control, through modern Automatic Train Protection (ATP) systems, to full ERTMS/ETCS deployments and traffic management platforms. Recent steps have focused on communication-based train control (CBTC) in metros, ETCS rollout on mainlines, and the integration of predictive maintenance and data analytics tools. Projects such as national ETCS rollouts, the UK East Coast Digital Programme, and research initiatives like ATO-Cargo exemplify a shift from isolated automation pilots to system-wide modernization that combines ATO, interoperability standards (TSIs), and remote supervision concepts \cite{era_tsi, dlr2025atocargo}. 

\subsection{System Overview}
A section to go threw the system as it stands.

\subsubsection{Components}
A overview of the components in use.
Also mentioning alternatives to the ones we have in use.