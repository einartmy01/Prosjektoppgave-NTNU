\section{Introduction}
The chapter of Introduction will contain text about purpose, research questions and a small bulk about history and the level of digitalisation we have reached right now.

\subsection{Purpose}

The purpose of this paper is to establish a structured foundation for understanding, measuring, and evaluating latency in remote train operation (RTO) systems. As the railway sector transitions toward higher Grades of Automation (GoA), particularly GoA~3 and GoA~4, remote control becomes an essential operational component. Previous research highlights that RTO is a required for safely deploying and validating autonomous capabilities, as it enables controlled testing and secure intervention when automated functions cannot perform as intended \cite{jürgensen2025rcrailvehicles}. In this context, RTO functions as a mandatory fallback layer within the ATO architecture, ensuring continuity of safe operation during degraded modes or for unexpected track conditions \cite{kozarevic2025}.

A important requirement for such fallback control is the availability of reliable low-latency video transmission. Studies on remote train supervision emphasise that successful human intervention depends heavily on the stability and responsiveness of video streams when the physical control room is replaced by a remote workstation \cite{mejias2024}. Industry demonstrations show that remote operation can supports practical tasks such as shunting, depot movements, and vehicle preparation and thereby improving efficiency while maintaining safety \cite{fp2r2dato2024d41_2}. At the same time, the downgraded mode scenarios described in deliveries from Europe's Rail \cite{fp2r2dato2023d5_4} show how responsibility will transition between automated systems and human operators. And it is therefore important to identify the operational limits imposed by latency, uncertainty, and reduced information quality.

An equally important aspect is understanding how latency affects human perception and reaction time. Human-factors research provides insight into how video quality, bitrate, and delay influence a remote operator's ability to detect signals, interpret scene dynamics, and perform corrective actions under time constraints. These findings contribute to defining minimum perceptual requirements for safe fallback operation and support the formulation of latency thresholds for high automation environments \cite{brandenburger2023vqreactiontimertc}.


\subsubsection{Research Questions}
To achieve the purpose of this paper, the research questions listed below was created to help. 

\begin{enumerate}
    %\item RQ1: How does other industries set threshold for latency?
    \item RQ1: How is latency evaluated for railway and other industries? 
    \item RQ2: What parameters influence latency in remote train operation systems?
    \item RQ3: How can we test different video streaming protocols and their impact on real-time performance of remote train operation systems?
\end{enumerate}

\subsection{Digitalisation in Railway}

Railway digitalization has progressed from early computer-assisted signalling and centralized traffic control systems to modern Automatic Train Protection (ATP), ERTMS/ETCS deployments, and integrated traffic management platforms. Recent developments include communication-based train control (CBTC) in metros, ETCS rollout on mainlines, and the integration of predictive maintenance and data analytics tools. Projects such as national ETCS rollouts, the UK East Coast Digital Programme, and research initiatives like ATO-Cargo exemplify a shift from isolated automation pilots to system wide modernization that combines ATO, interoperability standards (TSIs), and remote supervision concepts \cite{era_tsi, dlr2025atocargo}. 

The current digitalisation also shapes how future railway operations are organised. Projects across Europe show that automation and improved connectivity all depend on reliable digital systems that support both automated driving and human supervision. Remote control is becoming a standard element in this transition because it allows operators to intervene safely when automated functions cannot handle a situation. It also supports new operational concepts such as remote supervision of rural lines and more efficient traffic management \cite{jürgensen2025rcrailvehicles}. At the same time, the move toward FRMCS and 5G aims to replace GSM-R and provide the higher bandwidth and lower delay needed for real-time video and control during remote operation \cite{mejias2024}.

Digitalisation also increases the system's dependency on stable communication and sensor data. Research shows that issues such as reduced bandwidth, sensor faults, or missing information can directly affect whether remote operation is possible, making clear procedures for degraded modes essential \cite{fp2r2dato2023d5_4}. Human factor studies further underline that the quality of digital perception. Especially video streams must be high enough to replace traditional visibility when operators supervise the train remotely \cite{brandenburger2023vqreactiontimertc}.

\begin{comment}
\subsection{System Overview}
A section to go threw the system as it stands.

\subsubsection{Components}
A overview of the components in use.
Also mentioning alternatives to the ones we have in use.
\end{comment}