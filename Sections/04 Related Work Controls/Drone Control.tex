\subsection{Drone Control}
Drones are only remotely operated as it is a unmanned aerial vehicle (UAV) / unmanned aircraft system (UAS). Making research on latency or remote operating extremely relatable to RTO. Even though drones are controlled very differently and operated in a completely different environment, there is much to learn from the research done for remote operation.

\subsubsection{González 2023}

González et al. \cite{gonzález2023} study how live video for first person view (FPV) drone control behaves over LTE and WiFi networks. The purpose is to assemble a full test and measure how telemetry, control, and video traffic perform under different connection setups. They build a quad-rotor platform, connect it to LTE through an interface module, and test three schemes: LTE server-based, LTE direct, and WiFi direct. The authors record packet-level data, video quality, and latency to understand how each aspect influences FPV usability. They then define a QoE model combining VMAF video scores with frame-delay measurements, and use controlled packet loss and delay to evaluate how image quality and latency affect operator experience. This provides a structured method for assessing FPV performance in cellular networks. 

\subsubsection{Larsen 2022}

Larsen et al. \cite{larsen2022} examine whether a 5G network can support safe drone control when very low latency is required. The report looks at how delay is created as data travels from a drone, through the radio link, into the 5G transport network, and finally to a ground control station. To understand this, the authors break the network into segments and study how distance, number of switches, message size, and link speed each add delay. They then use these factors to calculate how far a drone can realistically fly from the control station while still meeting strict real-time requirements. The work also compares autonomous flight and manual remote control to show how their communication needs are different. In general, the report provides a framework for deciding how strong and fast the network must be, depending on how far drones are expected to operate and how much data they send.

\subsubsection{Böhmer 2020}

The report done by Böhmer et al. \cite{böhmer2020} investigate how open communication stacks can be used for remote drone control while still achieving low and predictable latency. The authors build a full test system around the Crazyflie nano-drone by extending it with a Raspberry Pi to enable WiFi communication and by developing a bridge that forwards control messages between the drone and a ground station. They implement two different bridges, one in Python and one in Rust before they analyse how each contributes to overall delay in the control loop. Using detailed timestamp logging and cross-layer analysis, the paper breaks down latency into processing steps on the Raspberry Pi, UART transfer times, and the wireless path. This allows them to compare their open protocol stack with the drone's original proprietary radio link and determine whether the open solution is viable for control tasks.



