\subsection{Train Control}
As showed in Table \ref{tab:vehicleSearch} there is only a handful of papers connected to Railway control compared to the rest. Therefore have I selected a few of the papers discussing latency, ATO and remote control of exactly trains.

\subsubsection{ATO-Cargo Project}
The ATO-Cargo project, led by the German Aerospace Center (DLR) in cooperation with DB Cargo AG, Digitale Schiene Deutschland (DSD), and ProRail B.V., focuses on developing and testing highly automated technologies for freight trains. The goal is to enhance rail freight efficiency by optimizing speed profiles, improving route utilization, and increasing competitiveness with road transport \cite{dlr2025atocargo}.

A key component of the project is the integration of an Automatic Train Operation (ATO) unit on locomotives in combination with the European Train Control System (ETCS) Level 2. This setup allows for real-time automation while maintaining human oversight. In case of system malfunctions or degraded operation, human operators at a Remote Supervision and Control Centre (RSC) can take over tasks such as remote monitoring, diagnosis, and manual control. 

The project also emphasizes human factors engineering, ensuring that the RSC is ergonomically designed for operator efficiency and safety. For this project, researchers have employ virtual reality tools to simulate realistic control room environments and train personnel for future remote supervision tasks. Tests are being conducted on Betuweroute, a freight only railway, linking Rotterdam and the Ruhr region to validate the technical and operational readiness of this automation concept. The ultimate goal is to establish a European reference model for automated freight train operation \cite{dlr2025atocargo}.

\subsubsection{Jürgensen 2025}

The report by Jürgensen focuses on developing a functional remote control system for autonomous rail vehicles, using the REAKTOR project as the primary use case. Its main purpose is to enable safe human intervention during intermediate automation stages and to act as a fallback solution during fully autonomous GoA~4 operations. The report frames remote control as a crucial safety mechanism, both during prototype development and as an operational safeguard in future autonomous traffic scenarios \cite{jürgensen2025rcrailvehicles}.

The work develops a complete client-server architecture for remote driving, consisting of a Web Client for user interaction, a server coordinating data exchange, and a vehicle-side component interfacing with the autonomous controller. The system enables control over speed, and braking while providing a livestream from onboard cameras. The thesis additionally evaluates the type of cellular network needed for stable operations and includes a case study of network quality along the Malente - Lütjenburg track \cite{jürgensen2025rcrailvehicles}.

Overall, the paper concludes that a remote control system is essential for autonomous rail, but its reliable operation depends heavily on network latency and coverage. Full-scale deployment requires improved cellular infrastructure, ideally 5G, to ensure continuous low-latency transmission \cite{jürgensen2025rcrailvehicles}.

\subsubsection{Kozarevic 2025}
The report by Kozarević aims to develop a remote control fallback system for autonomous rail vehicles, with a strong focus on identifying what functionality a remote operator must have in order to safely intervene when autonomy becomes unavailable. The main purpose of the work is to ensure that a human operator can reliably take over control in a downgraded mode. The thesis explores this by narrowing down required features and by analysing user needs, system safety conditions, and operational constraints for future automated railway environments.\cite{kozarevic2025}

The work proceeds by starting from a broad set of functions that a driver normally performs and reducing them to the subset that is strictly necessary for remote operation. The thesis focuses on determining which controls, displays, and sensory feedback elements a remote fallback system must include and evaluates the role of human factors, describing how situational awareness degrades when control is exercised remotely. The thesis ends talking about how remote fallback operation is feasible but requires a carefully reduced function set tailored to remote conditions. \cite{kozarevic2025}

\subsubsection{Mejías 2024}
The paper by Mejías et al. \cite{mejias2024} examines how video streaming can support remote train driving, focusing on keeping latency low enough for safe operation. The authors compare RTSP and WebRTC in a test setup that imitates an onboard camera system sending video over a 5G network to a remote operator. A key part of the work is a method for measuring true end-to-end latency by embedding timestamps directly into each frame. They also develop a simple adaptive bitrate mechanism that reacts to jitter and packet loss to keep the video usable under degrading network conditions. Tests in two laboratory setups show how latency grows as bandwidth drops and how bitrate adaptation delays freezing and pixelation under poor network conditions.

\subsubsection{FP2R2Dato, EuropesRail 2024 D41.2}
This deliverable D41.2 from Europe's Rail and FP2R2Dato project presents the testing results and assessments from the first remote driving demo for tramways. 
Its main purpose is to document how the system behaved during the initial demonstration phase, where auxiliary, static, and dynamic tests were performed on a modified SL18 tram in Oslo. The work focuses on verifying whether the new remote driving and command functions were correctly integrated, safe and aligned with the defined use cases. All tests were done with a structured procedure with reports, and acceptance criteria where each test case was evaluated as compliant or not. The document also includes KPI assessments such as image quality, latency, operator experience, availability, and productivity impact. Overall, the results show that the demonstrator reached the target and that the remote driving solution works reliably across the tested scenarios \cite{fp2r2dato2024d41_2}.

\subsubsection{FP2R2Dato, EuropesRail 2023 D5.4 - Chapter 12}
Chapter 12 in the deliverable D5.4 describes how remote driving should operate when faults occur in the remote control chain. The purpose is to define use cases for degraded remote control conditions, such as poor visibility, weak communication links, sensor failures, or loss of track information. These use cases outline how a remote driver should react and what operational limits apply when the supporting systems no longer provide full-quality information. The chapter does not aim to solve the failures but to describe expected driver actions and system behaviour in each degraded mode \cite{fp2r2dato2023d5_4}.

The deliverable also describes allowed latency by comparing it to a previous drone test, where it does calculations of speed and creates a allowed speed limit depending on latency experienced \cite{fp2r2dato2023d5_4}.

\subsubsection{Brandernburger 2023}
Brandenburger et al. \cite{brandenburger2023vqreactiontimertc} present exploratory studies on how video quality affects a remote driver's ability to perceive information and react in time. The purpose of the work is to understand how bitrate, frame rate, and stimulus type influence perception accuracy and reaction speed when supervising trains through video feeds. The authors ran two small-scale studies where participants identified either light signals or distance markers in short video clips of varying quality. Results showed a clear trend: higher bitrate generally improved accuracy and reduced response time, while frame rate showed no meaningful effect. Distance markers were consistently identified faster and more reliably than light signals. Overall, the work provides early evidence that bitrate is the more critical parameter for supporting remote train operation.


