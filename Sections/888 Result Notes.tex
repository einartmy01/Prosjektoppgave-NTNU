\begin{comment}
%%%%%%%%%%%%%%%%%%%%%%%%%%%%%%%%%%%%%%%%%%%%%%%%%%%%%%%%%%%%%%%
\cite{brandenburger2023vqreactiontimertc}
- Evaluates latency by performance, (human) IRL

Test of human factors and realiable communcations via 5g
"Limited literature on:" 
"Positive effect of bitrate on quaility", "Stalling has worse effect on QoExperience, if bitrate is higher"
"Higher frame rate not linked to information assimiliation, but increased user enjoyment"

Tested with three different levels of bitrate
1, 6, 24 Mbps
5, 15, 25 FPS
Stimulus
Light signal, Distance marker

MeasurementsResponce accuracyResponce speed

Study 1:
Higher bitrate -> faster answears, more correct
Higher FPS -> same speed on answear, same amount of correct
Study 2:
Higher bitrate -> same speed on answears, more correct
Higher FPS -> same speed on answear, same amount of correct

Bitrate is more important (source 5 in PP: https://dl.acm.org/doi/abs/10.1145/2072298.2072351)
Stimulus type:
Distance marker signs where answeared faster and more correct than light signals
5000 -> 3000 speed, and 0.51 -> 0.58 and 0.61 -> 0.69

This is promosing new with the implementation of ERTMS as the only input stimulus in signalling threw camera and video stream will be signs as the remote control operatior will get the ETCS directly in the control room. 
After certain bitrate less helpful
%%%%%%%%%%%%%%%%%%%%%%%%%%%%%%%%%%%%
\cite{brunnström2020}
- Evaluate latency on performance, (Humans) Sim
To study QoE. 
- VR.
- 270 angle HMD video threw 4 cameras on crane.
- Joystick, 10, 20, 50, 100, 200, 400, 800 -> 80, 90, 100, 130, 180, 280, 480, 880

Comfort of the subject was not affected by joystick delay, but the display delayed had an negative effect of Comfort quality. Why this could be discussed alot. Might have to do with VR and a more moving image. 
%%%%%%%%%%%%%%%%%%%%%%%%%%%%%%




%%%%%%%%%%%%%%%%%%%%%%%%%%%%%%%%%%%%%%%%%%%%%%%%%%%%%%%
Its important to include how humans act in different scenarios, as regardless of the systems, the remote operator will always be a constant high latency of at least 500 - 1000 ms \cite{ouden2022}. And how they react to a possible system latency will also be crucial for the full remote operation.
%%%%%%%%%%%%%%%%%%%%%%%%%%%%%%%%%%%%%%%%%%%%%%%%%%%%%%%%%%%%%%%%%%%%%%%%%%%%%%%%%
%%%%%%%%%%%%%%%%%%%%%%%%%%%%%%%%%%%%%%%

\cite{gonzález2023}
- Evaluation tools for latency, 
- People tested

URLLC
4K quality res 1080 x 720, at 30 FPS
H.264 encoded

LTE server, LTE direct, WiFi -> avg packet delay =  500 ms, 42 ms 4 ms %Er et bilde i filutforsker hvis trengs
Can activate low latency mode. 
Connection requirments of 100 ms for video streaming, set by 3GPP TS 22.829 for unmanned aerial vehicles
Found added latency of around 180 start causing lower MOS / QoE, and steady deteriation making it 1, lowest grade at around 460 ms
concludes with a e2e of 250 ms is vaiable for service usability.
%%%%%%%%%%%%%%%%%%%%%%%%%%%%%%%%%%%%%%%%%%%%%%%%%%%%%%%%%
\cite{jernberg2024}
"However, it also seems to be the case that remote operators adapt to the circumstances they find themselves in; for example, they drive with a safety margin reducing risks to their personal chosen limit and do not override the barriers of their own choice."

"The reaction time in H1 (a car cutting in into the ego lane) increased more, for each latency condition, than the offset in time that the manipulation created. This suggests that even an unperceived increase in latency (based on subjective ratings) is affecting the participants in a way that makes them less observant of their surroundings. Combined with a self-reported decline in performance and control as the latency increases, a conclusion could be that the added mental workload is turning the primary driving task into a distraction in itself, by forcing the operator to devote an unusual amount of attention just to maintain speed and lane position."
\end{comment}
\begin{comment}
In the case of Kozarevic who reference Jernberg 2024 adn their findings of. Jernberg goes into details of the hazards and proxy hazards that the driver were facing. As mentioned in Section 4 "Related Work" \cite{jernberg2024}
How in control of the vehicle were you during the drive (1-5)
Baseline 3.7, 100ms 3.5, 200ms 2.9


The most referenced remote control latency effect paper, Neumeier's paper \textit{Teleoperation: The Holy Grail to Solve Problems of Automated Driving? Sure, but Latency Matters} \cite{neumeier2019}, talks about how participant leave the car lane significantly more with higher latency, even tho with stable high latency. However, as mentioned Neumeier got no differences for whatever latency\cite{neumeier2019}
Scenarios, was driving with turns, and one parking. No hazards except latency. 

%%%%%%%%%%%%%%%%%%%%%%%%%%%%%%
\cite{kozarevic2025}
- Finds actually latency in project
- Uses Car / Drone threshold to evaluate
Talks about the dangers of latency in high speed vehicles. Comparing it to drone operations.
Reference drone and says 100 ms
Reference Chen?, and Neumeier, 170 ms and 300 ms have minimal impact on remote operators.
%%%%%%%%%%%%%%%%%%%%%%%%%%%%%%%%%%%%%%%%%%%%%%%%%%%%%%


\cite{jürgensen2025rcrailvehicles}
- Finds actually latency in project
- Uses Car threshold to evaluate

Is a project in "Remote Control for Rail Vehicles" where they test a remote control train for a short track from "X" to "X" in Germany. Here they do "this" and found "that".

Says that at 300ms, you get loss of performance, and at 1000ms the delay becomes unfeasible.
And references: "Design and Evaluation of Remote Driving Architecture on 4G and 5G Mobile Networks" (Ouden, 2022) \cite{ouden2022}
Which references (Lane, 2002) and (Neumeier, 2019)
%%%%%%%%%%%%%%%%%%%%%%%%%%%%%%%%%%%%%%%%

D41.2 - Testing reports \& assessment 
Results of the remote driving of tramways demonstrator. 
\cite{fp2r2dato2024d41_2}
- Finds actually latency
\cite{fp2r2dato2024d41_2} % Need more information about bitsize + other
D41.2 - Testing reports \& assessment 
Results of the remote driving of tramways demonstrator. 
Image latency 
- G2G, (capturing processing, compression, transmission, reception, decompressing, displaying)
- Oslo to Berlin
- Two atomic clocks on phones.
- Measured to be 340 - 380 ms (Always under 400 ms)

Image latency 
- G2G, (capturing processing, compression, transmission, reception, decompressing, displaying)
- Oslo to Berlin
- Two atomic clocks on phones.
- Measured to be 340 - 380 ms (Always under 400 ms)


Auxiliary circuit tests,
- Driver safety
- Remote wake up
- R driving loop
- R control commands
- R ...
Static functional tests,
- Start Tram, CERES,
- CERES do step 1.2.3\dots

Dynamis functional tests,
- CERES Local brakte test
- CERES Remote Brakte test 
- C Drive 5km/h 
- C Drive 100\% dont break max speed 
- C loose communication.
- C DSD brake sequence
- C Local driver break priority over remote.

Reaction Time,
reduction in time needed to perform spesific tram operations, fleet management and preparation, start-up and shut-down procedures, maintenance tasks, shunting. 
Time reduction in these processes are efficiency improvments.
%%%%%%%%%%%%%%%%%%%%%%%%%%%%%%%%%%%%%%%%%%%

\cite{ouden2022}
- Evaluates Latency by performance, (people) IRL  

- Evaluate Latency threshold on SIMulation

- 4G and 5G 
- 4 times 120 angle camera
- H.264
- Split latency up into Control and video and does
- min, mean, 95\%ile, max latency in ms

- Speed of 10, 20, 30 and 40km/h
100 manual test runs for benchmark
180 runs of RC with 4G
300 runs of RC with 5G 
One Trip Latency.” Every unit
was time synchronized with a GPS-PPP source.
Packets logged using tcpdump.


Includes a reference that results in at 300ms, you get loss of performance, and at 1000ms the delay becomes unfeasible.
%%%%%%%%%%%%%%%%%%%%%%%%%%%%%
\cite{jernberg2024}
- Evaluates Latency by performance, (people) Sim

- Voysys
- G2G
- Average delay of 88.8 ms and added conditions of +100 -> 188 ms and +200 -> 288 ms
- 50km/h and 70km/h (Try to keep speedlimits)


%%%%%%%%%%%%%%%%%%%%%%%%%%%%%%%%%%%%%%%%%%%%%%%%%%%%%%%%%%%%%
\end{comment}

\begin{comment} 
Very difficult from related works:
Real time video latency: \cite{kaknjo2018videolatency}
Here, the time of the visual event in front of
the camera is denoted as T1 and the time when the event
was detected on the receiving end as T2. The start of frame processing is denoted as T3.
\textit{TVL =(T1 - T1)-(T2 -T3)=T3 -T1}
%%%%%%%%%%%%%%%%%%%%%%%%%%%%%%%%%%%%%
\cite{larsen2022}
- Evaluation latency on perfomance, (tools) IRL
5G URLLC network
H.264

Le2e = Lprop agation + Lproc essing + Lser ialisation
Lprop = distance / v in medium
Lser = S datasize / R transmission rate

0.5 Mbps video rate in uplink and a 60 Hz update rate in downlink. Further, we assume that the higher quality video for inspection require 8 Mbps.
%%%%%%%%%%%%%%%%%%%%%%%%%%%%%%%%%
%%%%%%%%%%%%%%%%%%%%%%%%%%%%%%%%%
\cite{böhmer2020}
- Evaluation latency on perfomance, (tools) IRL

Predictably Reliable Real-time Transport (PRRT) protocol [A. Schmidt, “Cross-layer latency-aware and -predictable data communication 2019]
The Crazyflie is controlled by Bitcraze's application layer protocol called Crazy Real-Time Protocol (CRTP)

Rasberry Pi including WiFi 2.4GHz due to Rasberry Pi constraints
timestamps by controller to drone: 
tp1 -> packet1 -> drone
tr1 <- response1 <- drone
tp2 -> packet2 -> drone
tr2 <- response2 <- drone

the Crazyradio communication path using the traditional radio link
the PRRT communication path with the Python bridge, and
the PRRT communication path with the Rust bridge.

IPT = tp2 - tp1 (Time between packets)
RTT = tr1 - tp1 (Round-trip time)

%%%%%%%%%%%%%%%%%%%%%%%%%%%%%%%%%%%%%%%%%%%%%%%%
\end{comment}

\begin{comment}
%%%%%%%%%%%%%%%%%%%%%%%%%%%%
\cite{mejias2024}
- Evaluate latency depending on parameters / tools
Compare:
- RTSP
- WebRTC Web Real time communication
as their Real-time Transport Protocol (RTP) protocol.
- E2E
- H.264

Methodology for latency measurement.
Network Time Protocol (NTP) is necessary to synchronize sender and receiver.
Server obtains the TS1 when image is captured. Adds it to the RTP packets generated after the encoding. Player retrieves the timestamp (TS1) from the RTP packets and compares it with the current time TS2 when the image is beeing displayed. To do this, you must retrive it from the package before the decoder and comapre it with the image comming out of the decoder. 

1 Capture: the camera captures an image together with the timestamp. The timestamp is added to the metadata of the image.
2 Encoding: the image is encoded into a H.264 bit stream. The metadata is maintained unaltered along the encoding process.
3 Encapsulation: the H.264 video stream is encapsulated into RTP payload. The capture timestamp is extracted from the metadata and added to the RTP header. For this, it is required both the RTP standard header and its RFC 8286 extension.
4 Sender: RTP packets are sent on the communication channel. In the case of RTSP, the player opens a connection with the sender. For WebRTC, a negotiation between the sender and receiver is performed through the signaling server to determine the communication route.

The player receives the RTP packets through RTSP or WebRTC and calculates the latency:
1 Receiver: it receives the RTP packets through the channel established with the media server.
2 Decapsulation: the original H.264 content is extracted from the RTP payload. In addition, the timestamp contained in the RTP header is extracted and added as metadata of the H.264 content.
3 Decoding: the H.264 content is decoded to retrieve the uncompressed image. The metadata is maintained unaltered along the decoding process.
4 Displaying: the image is displayed. Moreover, the timestamp is extracted from the metadata and subtracted from the current time to obtain the End-to-End latency. This is shown to the remote driver, who will consider it during the operations.

BITRATE 
Change bitrate regarding quality of output (jitter or packet loss). Bitrate vary between 5Mbps, 3.5Mbps, 2Mbps.
A change of 2\% packet loss and 500Hz / 1000Hz jitter.

implementation.
GStreamer framework. (Open source)
Pylon source from Basler element that capture of camera images and timestamps.
H.264 NVidia en/de coder. Provided by NVIDIA graphics cards. It is the key to enable bitrate adaption.
RTP H.264 pay/depay. For packaging encoded video signal into RTP packets, and RTP includes timestamp in header.

WebRTCbin. Allows communication via WebRTC, peer2peer, must connect to signaling server responsible for negotiation.
RTSP server/client. Manage connection and send/receive data.

Camera, Media server on Jetson Xavier (Either WebRTC or RTSP), Network equipment (switch or laptop simulating a router, allowing to evaluate against bandwith degradation). Computer as player and reciver.

Results
E2E, time after capture to the time before display
S2S, time in front of camera to time displayed on player image 

Measurments for each camera and alternating available bandwidth.
When enough bandwidth results in 150 ms S2S and <75 ms E2E 
Bandwidth of <=10 results in 570 ms - 1000 ms or pixelation freezing in both S2S and E2E

RTSP
Difference in S2S and E2E is approx 70 ms - 100 ms which is image capturing and displaying.

WebRTC is faster E2E but not S2S

RTSP with rate control
Allows the bandwidth to go past 7 Mbps that was issue before, although with high latency. 
Adjust itself back up again. Also we can see a shift in latency between latency when increased bitsize of video.  

%%%%%%%%%%%%%%%
\cite{kaknjo2018videolatency}
- Evaluate latency by perfomance, (different tools) IRL, 

- G2G
- Time stamps
- H.264
- MJPEG
- RTSP (Real Time Streaming protocol)
- TCP/UDP

Found MJPEG to be 300 ms lower latency than H.264. However it found H.264 to demand less bandwitdh 50-380Kbps as it compresses more enhensive than MJPEG 4.6-5Mbps.
Found detoriation in performance in latencies above 300 ms and increase in errors during control for latencies larger than 500 ms.
%%%%%%%%%%%%%%%%%%%%%%%%%%%%%%%
\subsubsection{Kang 2018}
\cite{kang2018}
- Evaluate latency by perfomance, (different tools) IRL, 

- 3 different resulutions (320x240, 640x480, 1280x960)
- 3 different bitrate (0.5Mbps, 1Mbps, 4Mbps)
- LTE and WiFi
- Video and camerea catching timestamps

%%%%%%%%%%%%%%%%%%%%%%%%%%%%%%%%%%%%%
\subsubsection{Larsen 2022}
\cite{larsen2022}
- Evaluation latency on perfomance, (tools) IRL

5G URLLC network
H.264

Le2e = Lprop agation + Lproc essing + Lser ialisation
Lprop = distance / v in medium
Lser = S datasize / R transmission rate

Le2e = nLproc + (n+1)Lser + Lprop + LQ 
n = switches along the network
n+1 = number of links
LQ = queing latency.

0.5 Mbps video rate in uplink and a 60 Hz update rate in downlink. Further, we assume that the higher quality video for inspection require 8 Mbps.
%%%%%%%%%%%%%%%%%%%%%%%%%%%%%%%%%
\cite{böhmer2020}
- Evaluation latency on perfomance, (tools) IRL

Predictably Reliable Real-time Transport (PRRT) protocol [A. Schmidt, “Cross-layer latency-aware and -predictable data communication 2019]
The Crazyflie is controlled by Bitcraze's application layer protocol called Crazy Real-Time Protocol (CRTP)

Rasberry Pi including WiFi 2.4GHz due to Rasberry Pi constraints
timestamps by controller to drone: 

the Crazyradio communication path using the traditional radio link
the PRRT communication path with the Python bridge, and
the PRRT communication path with the Rust bridge.

%%%%%%%%%%%%%%%%%%%%%%%%%%%%%%%%%%%%%%%%%%%%%%%%
\cite{gonzález2023}
- Evaluation tools for latency, 
- People tested

URLLC
4K quality res 1080 x 720, at 30 FPS
H.264 encoded

LTE server, LTE direct, WiFi -> avg packet delay =  500 ms, 42 ms 4 ms %Er et bilde i filutforsker hvis trengs
Can activate low latency mode. 
Connection requirments of 100 ms for video streaming, set by 3GPP TS 22.829 for unmanned aerial vehicles
Found added latency of around 180 start causing lower MOS / QoE, and steady deteriation making it 1, lowest grade at around 460 ms
concludes with a e2e of 250 ms is vaiable for service usability.

\end{comment}
