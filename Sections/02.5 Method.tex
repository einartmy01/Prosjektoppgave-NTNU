\section{Method}

\subsection{Research Methodology}
The following sections detail the methodological approach and structured review process used to address the research questions of this study. This methodology is designed to ensure a robust foundation for the evaluation of latency for remote train operations.

As of a scoping study described by Arksey \cite{arksey2005scoping} the process followed the steps of deciding research questions before moving over to identifying relevant studies. The research questions has been slightly modify to adapt to the information available and what was discover during the research phase. All the papers and studies was also selected before charting data, comparing, summarizing and reporting on the results. As mentioned in \cite{arksey2005scoping}, one point of a scope study is to identify research gaps in the existing literature. Which was very useful and ended up being most of the focus for this paper.

This study uses a Mixed Methods approach. Which includes combining two types of information: qualitative (ideas, experiences, and opinions) and quantitative (numbers and statistics). This method is chosen because using both together gives us a complete picture, where relying on only one type of data would not be enough \cite{creswell2014researchdesign}.

The study utilizes convergent parallel design for this. Which entails collecting and analysing the numbers and the opinions at the same time. Then comparing the results from both to see how they fit together and explain our final findings \cite{george2021mixedmethods}. The results from technical performance testing (quantitative) are thus validated and enriched by the practical feedback received from user trials (qualitative), leading to more actionable conclusions.

\subsection{Literature Review Strategy}
The literature review was conducted to establish a comprehensive theoretical and evidential basis for the research. The process was guided by the principle of Evidence-Based Standards to ensure methodological rigour and focus \cite{lim2008evidencebased}.

\subsubsection{Search Strategy and Databases}
A multi-platform search approach was utilized to retrieve a wide array of high-quality sources. The primary databases included:

\begin{itemize}
    \item Scopus: For retrieving peer-reviewed, high-impact scientific articles.
    \item Google Scholar: For broader academic and institutional literature.
\end{itemize}

While maintaining a strong reliance on peer-reviewed scientific material, relevant non-academic reports and industry publications were also considered to provide a comprehensive perspective, with all information traced to reliable sources. Especially theoretical part of the background research has benefited from non-academic sources.

\subsubsection{Inclusion Criteria}
To ensure the study is based on the most relevant and current information, specific criteria were applied to filter the search results. Given the rapid pace of technological change, a focus was placed on recent publications. Where modern papers have referenced to older papers, I have include some as to say and show where certain numbers are coming from and how they have influenced results.

\begin{table}[h]
    \centering
    \caption{Inclusion Criteria for Literature Review}
    \label{tab:inclusion_criteria}
    \begin{tabular}{|c|p{8cm}|}
        \hline
        ID & Inclusion Criteria (IC) \\
        \hline
        IC1 & Publication Date: Between 2020-2025 \\
        \hline
        IC2 & Language: Written in English \\
        \hline
        IC3 & Document Type: Primarily "Article", "Conference Paper" or "Thesis"\\
        \hline
    \end{tabular}
\end{table}

I did not include a specific keyword criteria because of the exploratory nature of the research questions. Instead, broad search terms were used initially, with relevance determined through title and abstract screening against the inclusion criteria.

\subsubsection{Search Queries and Refinement}
Initial broad queries were executed and subsequently refined to focus on specific research gaps, such as the intersection of video communication and system latency. For example, Table \ref{tab:genericSearch} illustrates a query targeting the core technological elements of the study.

\begin{table}[H]
    \centering
    \caption{Summary of Refined Search Queries. \label{tab:genericSearch}}
    \begin{tabularx}{\textwidth}{|X|c|c|}
        \hline
        Search Query & Initial Hits & Filtered Hits\\
        \hline
        "Remote" "Train operation" OR "Train control"               & 12 900 & 5 380 \\
        \hline
        "Automatic" "Train operation" OR "Train control"            & 17 600 & 11 900 \\
        \hline
        "Vehicle" "streaming" "protocols" "real-time" "performance" & 36 800 & 17 400 \\
        \hline
        "Ethics" "Remote" Control" "Vehicle"                        & 298 000 & 34 200 \\
        \hline
        "Cybersecurity" "Remote" "Control" "Vehicle"                & 42 600 & 21 600 \\
        \hline
        "Latency awareness" "Remote" "Control"                      & 460  & 294 \\
        \hline
    \end{tabularx}
\end{table}

For more detailed searches and for finding reports to compare with, the following queries in Table \ref{tab:vehicleSearch} stands for the majority of papers.

\begin{table}[H]
    \centering
    \caption{Summary of Refined Search Queries.}
    \label{tab:vehicleSearch}
    \begin{tabularx}{\textwidth}{|X|c|c|}
        \hline
        Search Query & Initial Hits & Filtered Hits\\
        \hline
        "Remote" "Latency" "Operation" OR "Control"                         & 841 000 & 83 000 \\
        \hline
        "Remote" "Latency" "Operation" OR "Control" "Railway"               & 11 800 & 6 580 \\
        \hline
        "Remote" "Latency" "Threshold" "Operation" OR "Control"             & 245 000 & 33 700 \\
        \hline
        "Remote" "Latency" "Threshold" "Operation" OR "Control" "Railway"   & 4 350  & 2 530 \\
        \hline
    \end{tabularx}
\end{table}

The results of the query clearly show a minor part of the total remote control "community" specialize in railway operation. There is also possible to see a trend showing that it has become increasingly more popular in recent years as around half of the papers are from the last 5 years.

To be able to choose from multiple thousands papers down to the ones chosen, a filtration method has been used as you can see in Figure \ref{fig:prisma}. Numbers are based on the example of 33 700 for \textit{"Remote" "Latency" "Threshold" "Operation" OR "Control"}. Where "Headline filtration" was an estimate on how relevant it appears to be for this research based on headline, amount of citations and year produce where newer papers was judges easily on citations. "Abstract reading" filtered looking for relevant context of the papers. "Full reading" confirmed value for the review and research, looking for useful information that could contribute to this scoping review. Finally "Selection" was papers adding new perspective not already discussed by another paper, or papers reconfirming other papers.
\begin{figure}[H]
    \centering
    \includegraphics[width=0.9\textwidth]{Images/Prisma.png}
    \caption[PRISMA Flow Chart]{PRISMA flow chart}
    \label{fig:prisma}
    \source{Own product}
  \end{figure}
\subsection{Supporting Tools}


Several digital tools as mentioned below were used for this paper. The main purpose of the usage was to enhance clarity, assisting in latex, and in general increase quality. Specifics include writing reference list in correct format. Structuring sentences and paragraphs. Checking for typos and grammatical errors.

\begin{itemize}
    \item OpenAI's ChatGPT: Employed as a helpful resource for LaTeX formatting suggestions, generating structured content (tables and lists), translating between Norwegian and English, and reviewing text for synonyms and restructuring ideas to ensure arguments were effectively communicated.
    \item Google Gemini: Used for similar fields as ChatGPT, but also for general proofreading, and refining sentence structure and tone to maintain a high standard of academic writing.
    \item Visual Studio Code: was used as the main LaTeX editor because of its powerful extensions for LaTeX support, syntax highlighting, and version control integration.
\end{itemize}

\subsection{Validity}
In "Reliability and Validity in a Nutshell", Bannigan and Watson define validity as the extent to which a measurement accurately measures what it is intended to measure \cite{bannigan2009raliabilityvalidity}. When identifying relevant papers, I have tried different setups and keywords to make sure they give the wanted output. An example is to use just "Train" instead of "Railway" will give a lot more results as it will includes all reports talking about training personal or computers. Or if you restrict to "Train Control" the results drop significantly, loosing important papers for this topic. It is difficult to determine if the aspects I have focused on is the best and correct way to compare and find a good solution. However, as I will also discuss in Chapter \ref{sec:ResultsDiscussion} "Results and Discussion", many of the other vehicle industries have decided to focus on the same aspects. 

\subsection{Reliability}
Bannigan and Watson describe reliability as the degree to which a measurement is consistent and dependable \cite{bannigan2009raliabilityvalidity}. In this report, reliability is addressed by drawing on a wide range of studies from different industries before making comparisons. This approach is intended to increase consistency and dependability, such that similar results could be achieved if the same research were conducted by others.
Based on the number of papers reviewed and included, this report aims to identify general patterns related to latency and threshold. These patterns are expected to be replicable, even if future studies use different source materials. As this is a broad examination of the topic, the overall findings should therefore be reproducible. In contrast, conclusions related to the testing and evaluation of tools and protocols may vary between researchers. These assessments rely more heavily on specific test scenarios and individual examples, which can reasonably lead to different interpretations and outcomes.