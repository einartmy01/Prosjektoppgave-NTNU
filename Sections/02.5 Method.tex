\section{Method}

\subsection{Research Methodology}
The following sections detail the methodological approach and structured review process used to address the research questions of this study. This methodology is designed to ensure a robust foundation for the evaluation of latency for remote train operations.

This study uses a Mixed Methods approach. Which includes combining two types of information: qualitative (ideas, experiences, and opinions) and quantitative (numbers and statistics). This method is chosen because using both together gives us a complete picture, where relying on only one type of data would not be enough \cite{creswell2014researchdesign}.

The study utelizes convergent parallel design for this. Which entails collecting and analyzing the numbers and the opinions at the same time. Then comparering the results from both to see how they fit together and explain our final findings \cite{george2021mixedmethods}. The results from technical performance testing (quantitative) are thus validated and enriched by the practical feedback received from user trials (qualitative), leading to more actionable conclusions.

\subsection{Literature Review Strategy}
The literature review was conducted to establish a comprehensive theoretical and evidential basis for the research. The process was guided by the principle of Evidence-Based Standards to ensure methodological rigor and focus \cite{lim2008evidencebased}.

\subsubsection{Search Strategy and Databases}
A multi-platform search approach was utilized to retrieve a wide array of high-quality sources. The primary databases included:

\begin{itemize}
    \item Scopus: For retrieving peer-reviewed, high-impact scientific articles.
    \item Google Scholar: For broader academic and institutional literature.
    \item \textit{Andre kilder/databaser}
\end{itemize}

While maintaining a strong reliance on peer-reviewed scientific material, relevant non-academic reports and industry publications were also considered to provide a comprehensive perspective, with all information traced to reliable sources. Espacially theoretical part of the background research has benefitted from non-academic sources.

\subsubsection{Inclusion Criteria}
To ensure the study is based on the most relevant and current information, specific criteria were applied to filter the search results. Given the rapid pace of technological change, a focus was placed on recent publications.

\begin{table}[h]
    \centering
    \caption{Inclusion Criteria for Literature Review}
    \label{tab:inclusion_criteria}
    \begin{tabular}{|c|p{8cm}|}
        \hline
        ID & Inclusion Criteria (IC) \\
        \hline
        IC1 & Publication Date: Between 2020-2025 \\
        \hline
        IC2 & Language: Written in English \\
        \hline
        IC3 & Document Type: Primarily "Article", "Conference Paper" or "Thesis"\\
        \hline
    \end{tabular}
\end{table}

I did not include a spesific keyword criteria because of the exploratory nature of the research questions. Instead, broad search terms were used initially, with relevance determined through title and abstract screening against the inclusion criteria.

\subsubsection{Search Queries and Refinement}
Initial broad queries were executed and subsequently refined to focus on specific research gaps, such as the intersection of video communication and system latency. For example, Table \ref{tab:search_results} illustrates a query targeting the core technological elements of the study.

\begin{table}[h]
    \centering
    \caption{Summary of Refined Search Queries. \label{tab:search_results}}
    \begin{tabular}{|p{7cm}|c|c|}
        \hline
        Search Query & Initial Hits & Filtered Hits (IC Applied) \\
        \hline
        "Remote train operation"                    & XX & XX \\
        \hline
        "Remote train operation latency threshold"  & XX & XX \\
        \hline
        "Remote control latency threshold"          & XX & XX \\
        \hline
        "Automatic train operation"                 & XX & XX \\
        \hline
        "Video streaming protocols AND real-time performance" & XX & XX \\
        \hline
        "Ethics remote control"                     & XX & XX \\
        \hline
        ""Latency awareness" AND "Remote control""  & XX & XX \\
        \hline
    \end{tabular}
\end{table}

The analysis \textit{confirmed} a limited number of high-relevance articles specifically addressing the impact of video streaming protocols on real-time performance, thereby confirming a critical area for this research to address.

\subsection{Supporting Tools}
Several digital tools as mentioned below were used for this paper. The main purpose of the usage was to enhance clarity, assisting in latex, and in general increase quality. Spesifics include writing reference list in correct format. Structuring sentences and paragraphs. Checking for typos and grammatical errors.

\begin{itemize}
    \item OpenAI's ChatGPT: Employed as a helpful resource for LaTeX formatting suggestions, generating structured content (tables and lists), translating between Norwegian and English, and reviewing text for synonyms and restructuring ideas to ensure arguments were effectively communicated.
    \item Google Gemini: Used for simialar fields as ChatGPT, but also for general proofreading, and refining sentence structure and tone to maintain a high standard of academic writing.
    \item Visual Studio Code: was used as the main LaTeX editor because of its powerful extensions for LaTeX support, syntax highlighting, and version control integration.
    \textit{Other tools}
\end{itemize}



