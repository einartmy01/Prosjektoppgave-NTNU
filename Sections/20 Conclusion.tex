\section{Conclusion}
This study explored how latency can be measured and evaluated for remote train operation systems, and how acceptable performance thresholds can be established. By reviewing research across railway, car, drone and crane control, and by analysing human responses alongside technical measurements, the study provides a clearer picture of what aspects of latency matter most and how they should be assessed.

\textit{Conclusion needs to be drawn up to RQ and try to give a little closure of what the paper discussed. 
\begin{enumerate}
    \item RQ1: How is latency evaluated for railway and other industries? 
    \item RQ2: What parameters influence latency in remote train operation systems?
    \item RQ3: How can we test different video streaming protocols and their impact on real-time performance of remote train operation systems?
\end{enumerate}
}


\subsection{Evaluation of latency}
The findings show that latency evaluation in train operation might need different methods than in other industries. Remote operation research on cars, drones and cranes often have different tests requiring movements, such as driving slalom or avoiding hazards, that train operation do not acquire. Making many of the latency's found in the research of other industries invalid or irrelevant. Only a few of the completed test for evaluation the latency can be comparable, these contain the setup similar to what a railway operator might be able to do in different scenarios, like acceleration, keeping speed limits, deceleration and stops.
Still there is many ways RTO research can learn from these industries. By examining setups, experiment formats and attributes considered when evaluating latency there is possibilities to replicate or adjust for RTO research. 


\subsection{Important aspects of latency}
The study also identifies the parameters that most strongly influence latency in remote train operation, addressing RQ2. Technical factors include encoding and decoding time, bitrate, packet loss, jitter, fps and network technology used. By being aware of the parameters affecting the total latency and how they interact together there is possibilities for finding individual limitations, identifying obstacles and ensuring progress by adjusting focus to problematic parameters and results.


\subsection{Measuring latency}
Just as other remote operation studies measure latency, train operation can also divide latency into one-way delay, round-trip time and glass-to-glass delay, and rely on synchronized timestamping to obtain accurate end-to-end measurements. By doing this, train operations can explore different setups, tools and protocols. The best way to accomplish a detailed and specific research on latency is to have a standardised setup where interchanging hardware and protocols to compare each aspect of the full remote operation transaction. To be able to decide for an approach it is necessary to have a complete understanding. Since remote train operation is a project for expansion, the research needs to be detailed enough to be able to predict how extra trains, extra distance, and extra disturbances will effect the system.

Overall, the study shows that remote train operation faces similar latency challenges to other teleoperated systems but benefits from the more predictable environment of railway infrastructure. While the literature provides useful initial thresholds, dedicated studies with trained railway personnel are still needed to define precise limits for safe operation. Future work should therefore focus on controlled human remote control experiments dedicated for train control as well as extended comparison of protocols, encoding tools, available hardware and all parameters affecting latency.

\subsection{Future work}
- MJPEG vs H.264

- Offshore or other remote controlled vehicles that can shead light or contribute to remote control railway.

Different aspects of latency \textit{What is the difference}
Fixed, static v dynamic.

\begin{comment}
\subsection{Parameters for latency testing}
What parameters are we measuring. 
Latency, Jitter, Packet loss.

\subsection{Parameters for latency evaluation}
speed, distance, braking distance, reaction time, margin for error.

- Lateral Deviation
- Max Steering angle
- Out of Lane Ratio 
- Avg Speed 
- Acceleration 

\subsection{Calculations of parameters}

\end{comment}

\textit{Unfinished thoughts:}
Theory v. Practical test

Compare from the PreDraft of what expected result and hopes were, and discussing them with the information of what happend in the test and measurments.




Just as we have observed how different video streaming tools has been used, upgraded and replaced, we must anticipate that today's tools also will be upgraded. Not to mention the wast spectrum of different models, businesses and products on the market able to provide the video streaming service. 

As businesses try to improve their products, 

As discussed previously, there is a lot of different ways to set up a remote control operation. You can choose from many different protocols, specialised hardware, or opt for a developed solution from an outside business. 
And as time grows, more option are destined to also arrive as the market are not yet satisfied with today's level. If we are to achieve the requirements set up by 3GPP for remote control under GoA~3 GoA~4 \cite{3gppts22289}, new protocols, hardware or a different way to do to remote operation is necessary. However, as of now, these requirment, should as mentioned in the paper be used as target values. When performing latency test