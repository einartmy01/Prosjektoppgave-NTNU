\section{Conclusion}
This study explored how latency can be measured and evaluated for remote train operation systems, and how acceptable performance thresholds can be established. By reviewing research across railway, car, drone and crane control, and by analysing human responses alongside technical measurements, the study provides a clearer picture of what aspects of latency matter most and how they should be assessed.

\subsection{Evaluation of latency}
The findings show that latency evaluation in train operation might need different methods than in other industries. Remote operation research on cars, drones and cranes often have different tests requiring movements, such as driving slalom or avoiding hazards, that train operation do not acquire. Making many of the latency's found in the research of other industries invalid or irrelevant. Only a few of the completed test for evaluation the latency can be comparable, these contain the setup similar to what a railway operator might be able to do in different scenarios, like acceleration, keeping speed limits, deceleration and stops.
Which is supported in Chapter 5 \ref{sec:ResultsDiscussion} by showing that the most comparable remote operation scenarios such as controlled braking, maintaining speed, and responding to visual cues behave consistently across different vehicle types. Studies on car, drone, and crane control all indicate that human operators begin adjusting their behaviour once latency approaches 250 - 300 ms, even when they do not consciously perceive the delay. This aligns with the measured railway results, where G2G delays around 340 - 380 ms were still sufficient for stable supervision tasks. Still there is many ways RTO research can learn from these industries. By examining setups, experiment formats and attributes considered when evaluating latency there is possibilities to replicate or adjust for RTO research. Also, the findings emphasise that performance is more sensitive to variability in latency than to the absolute value alone. This distinction is important for RTO, where predictable communication may outweigh the pursuit of the lowest possible delay.

\subsection{Important aspects of latency}
This study identifies the parameters that most strongly influence latency in remote train operation, addressing RQ2. Technical factors include encoding and decoding time, bitrate, packet loss, jitter, fps and network technology used. By being aware of the parameters affecting the total latency and how they interact together there is possibilities for finding individual limitations, identifying obstacles and ensuring progress by adjusting focus to problematic parameters and results. The results also show that the influence of these parameters is cumulative rather than isolated. For example, human factor studies demonstrated that lower bitrate and unstable video conditions had a stronger negative impact on perception and reaction time than raw latency alone, even when delay remained within acceptable ranges. Similarly, protocol comparisons revealed that mechanisms such as rate control, jitter handling, and encoding efficiency can significantly improve operational stability without altering the network itself. These findings reinforce that latency in RTO is not a single constraint but an interaction between network behaviour, video integrity, and operator workload. As a result, controlling these surrounding parameters becomes just as important as reducing delay, especially in critical tasks where clear visual information and stable feedback loops are essential.


\subsection{Measuring latency}
Just as other remote operation studies measure latency, train operation can also divide latency into one-way delay, RTT and G2G delay, and rely on synchronized timestamping to obtain accurate E2E measurements. By doing this, train operations can explore different setups, tools and protocols. The best way to accomplish a detailed and specific research on latency is to have a standardised setup where interchanging hardware and protocols to compare each aspect of the full remote operation transaction. To be able to decide for an approach it is necessary to have a complete understanding. Since remote train operation is a project for expansion, the research needs to be detailed enough to be able to predict how extra trains, extra distance, and extra disturbances will effect the system.
The measurements presented in Chapter 5 illustrate why consistent methodology is necessary for meaningful evaluation. The reviewed studies used timestamp based G2G and E2E techniques to separate camera processing, encoding, transmission, and decoding. This exposed how much of the total delay originates from video handling rather than the network itself. An insight that was confirmed by both railway and non-railway experiments. The comparison of RTSP, WebRTC, MJPEG, and H.264 further showed that measurement results can shift substantially depending on the chosen toolchain, even when the same network is used. This demonstrates the need for a unified measurement framework in RTO, where different setups can be tested under repeatable conditions and compared on equal terms. Establishing such a framework will be essential for future threshold validation and for ensuring that operational limits remain robust as new hardware, codecs, and FRMCS based networks are introduced.

Overall, the study shows that remote train operation faces similar latency challenges to other teleoperated systems but benefits from the more predictable environment of railway infrastructure. While the literature provides useful initial thresholds, dedicated studies with trained railway personnel are still needed to define precise limits for safe operation. Future work should therefore focus on controlled human remote control experiments dedicated for train control as well as extended comparison of protocols, encoding tools, available hardware and all parameters affecting latency.

\subsection{Future work}
The results of this scoping study highlight several areas where further research is necessary to establish reliable operational limits and technical requirements for remote train operation. Although many insights can be transferred from car, drone, and crane operation, the discussion shows that railway specific validation is still missing. The next steps should therefore focus on structured experimentation, standardised measurement frameworks, and system evaluations under realistic conditions.

A first priority is to conduct remote controlled experiments with trained railway personnel. Current thresholds are largely borrowed from cars and UAV studies, whose task demands differ significantly from train operation. Dedicated experiments using realistic RTO scenarios such as supervised cruising, signal interpretation, degraded mode interventions, and braking at defined reference points are needed to determine where latency begins to influence railway specific performance. 

A second important step is to develop a standardised, repeatable measurement methodology for latency in RTO. Results from Chapter 5 \ref{sec:ResultsDiscussion} demonstrate that differences in tools, encoding pipelines, and measurement procedures can heavily influence reported latency values. A unified framework covering G2G delay, E2E system latency, jitter tolerance, and bandwidth degradation would make it possible to evaluate different technologies on equal terms. Such a framework should be designed with future FRMCS and 5G URLLC requirements in mind, ensuring compatibility with evolving communication systems.

Further work is also needed on protocol and codec evaluation under relevant conditions. While current tests compare RTSP, WebRTC, H.264, and MJPEG, deeper analysis is required to understand how these tools behave over long distances, under varying load, or in the presence of packet loss patterns typical for cellular networks along railway corridors. Adaptive bitrate strategies, jitter buffers, and redundant streaming could play an important role in maintaining stable operator perception, and should therefore be tested systematically.

Finally, the practical implementation of RTO will require integration studies across the full \\ ATO/ETCS/FRMCS stack. This includes latency contributions from onboard equipment, interlocking communication, server-based processing, and remote control interfaces. Understanding how these elements interact in a complete system is essential for determining whether the overall architecture can meet the target values proposed by 3GPP TS 22.289, and where additional optimisation is needed.

Together, these future research directions will provide the empirical foundation needed to validate latency thresholds, compare technical solutions, and support the safe deployment of remote train operation within the ERTMS ecosystem.


\begin{comment}
\textit{Unfinished thoughts:}
Theory v. Practical test

Compare from the PreDraft of what expected result and hopes were, and discussing them with the information of what happend in the test and measurments.




Just as we have observed how different video streaming tools has been used, upgraded and replaced, we must anticipate that today's tools also will be upgraded. Not to mention the wast spectrum of different models, businesses and products on the market able to provide the video streaming service. 

As businesses try to improve their products, 

As discussed previously, there is a lot of different ways to set up a remote control operation. You can choose from many different protocols, specialised hardware, or opt for a developed solution from an outside business. 
And as time grows, more option are destined to also arrive as the market are not yet satisfied with today's level. If we are to achieve the requirements set up by 3GPP for remote control under GoA~3 GoA~4 \cite{3gppts22289}, new protocols, hardware or a different way to do to remote operation is necessary. However, as of now, these requirment, should as mentioned in the paper be used as target values. When performing latency test
\end{comment}