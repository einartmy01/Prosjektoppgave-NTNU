\section*{Abstract}

Remote train operation is an essential function for higher Grades of Automation, serving both as a development tool and as a safety fallback when automated systems degrade or fail. A critical requirement for remote operation is low and predictable latency, particularly for video transmission and control. This paper presents a scoping review of how latency is evaluated, measured, and interpreted in remote train operation systems, with supporting insights drawn from related domains including car, drone, and crane remote operation. The study reviews existing railway research, frameworks, and demo projects before comparing them with established methods and findings from other industries where remote control has been in development longer. Particular emphasis is placed on latency measurement techniques such as glass-to-glass, end-to-end, and round-trip transmission delay, as well as on the influence of video encoding, streaming protocols, network technologies, and human factors. The results show that while railway environments benefit from more predictable operation conditions than road or aerial vehicles, current latency thresholds are largely inherited from other domains and lack specific validation for trains. Human performance is shown to be sensitive not only to complete latency but also to latency variability, video quality, and system stability. The paper concludes that dedicated railway experiments, standardized latency measurement frameworks, and systematic testing of video streaming protocols are necessary to define reliable operational limits and support the safe deployment of remote train operation within future systems.
%cspell:disable

----------------------- Norwegian -----------------------

Fjernstyring av tog spiller en viktig rolle i systemer med høy grad av automatisering. Løsningen brukes både i utvikling, test og som en sikkerhetsløsning hvor automatiserte funksjoner ikke fungerer som planlagt. For at fjernstyring skal være optimal, må forsinkelsen i kommunikasjonen være lav og stabil. Dette gjelder spesielt for overføring av video og styringskommandoer mellom tog og operatør. Denne artikkelen gir en oversikt over hvordan forsinkelse i kommunikasjon analyseres og vurderes i løsninger for fjernstyrt togdrift. Fremstillingen bygger på gjennomgang av eksisterende jernbane relatert forskning, supplementert med erfaringer fra andre domener der fjernstyring har vært i bruk over lengre tid, blant annet innen bil, drone og kranoperasjon. Målet er å finne hvilke metoder som brukes for å måle og analysere forsinkelse, og hvordan disse anvendes i ulike sammenhenger. Videre gjennomgås sentrale tilnærminger til måling av forsinkelse, som måling fra kamera til skjerm, samlet forsinkelse gjennom hele systemet og fram og tilbake i kommunikasjonssystemet. Artikkelen diskuterer også hvordan valg av videokoding, strømmeteknologi, nettverksløsninger og operatøren påvirker både den faktiske og den opplevde ytelsen. Gjennomgangen diskuterer hvordan togdrift foregår under mer stabile og forutsigbare forhold enn mange andre former for fjernstyrte kjøretøy, spesielt biler og droner. Likevel er mange av kravene angående forsinkelse som benyttes i dag hentet fra nettopp bil og drone, uten at de er tilstrekkelig dokumentert eller verifisert for bruk i jernbanesystemer. Resultatene indikerer også at operatørens prestasjon påvirkes av flere forhold enn bare total forsinkelse, blant annet variasjoner i forsinkelse, bildekvalitet og systemets stabilitet over tid. Til slutt konkluderes det med behovet for målrettede eksperimenter tilpasset jernbane, felles retningslinjer for hvordan forsinkelse skal måles, samt grundigere testing av strømmeløsninger for video. Slike tiltak er nødvendige for å kunne fastsette realistiske og sikre driftsgrenser og for å legge til rette for trygg innføring av fjernstyrt togdrift i fremtidige systemer.